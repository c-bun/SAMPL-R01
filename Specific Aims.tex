\documentclass[11pt]{article}
\usepackage[top=0.5in, bottom=0.5in, left=0.5in, right=0.5in]{geometry}
\usepackage{helvet}
\usepackage{url} % hypderref?
\usepackage{graphicx}
\graphicspath{{figures/}} % The figures are in a figures/ subdirectory.
\renewcommand{\familydefault}{\sfdefault}
\pagestyle{empty}
%\pagestyle{plain}

\usepackage{setspace}
\usepackage{microtype}

\usepackage{amsfonts}
\usepackage{amsmath}

\usepackage[normalem]{ulem} % for nci.bst

\usepackage{sidecap}
\usepackage[abs]{overpic}
\usepackage{wrapfig}

%\usepackage[round,authoryear]{natbib}
\usepackage{cite}
%\setlength{\bibsep}{0.00in}

\usepackage{hyperref}
\hypersetup{colorlinks=true, urlcolor=black, citecolor=black, linkcolor=black}

\usepackage[numbers,sort&compress]{natbib}

\newcommand{\doi}[1]{\href{http://dx.doi.org/#1}{doi:#1}}

\newcommand{\ac}[1]{{\sc \lowercase{#1}}}

\renewcommand{\baselinestretch}{.93}
%\renewcommand{\baselinestretch}{.90}
\usepackage{wrapfig}

\usepackage{bibspacing}
\setlength{\bibspacing}{\baselineskip}
\usepackage{bibentry}


\graphicspath{{figs/}}

\makeatletter

\newcommand{\captionfonts}{\footnotesize}

\makeatletter  % Allow the use of @ in command names
\long\def\@makecaption#1#2{%
  \vskip\abovecaptionskip
  \sbox\@tempboxa{{\captionfonts #1: #2}}%
  \ifdim \wd\@tempboxa >\hsize
    {\captionfonts #1. #2\par}
  \else
    \hbox to\hsize{\hfil\box\@tempboxa\hfil}%
  \fi
  \vskip\belowcaptionskip}
\makeatother

\renewcommand{\figurename}{Fig.}

% Page numbering.
%\pagestyle{plain}
%\pagenumbering{arabic}

\setlength{\abovecaptionskip}{-5pt}

\makeatother

\renewcommand{\refname}{Bibliography and References Cited}

\setlength{\parindent}{0pt} % Don't indent first line
%\setlength{\parskip}{1ex plus 0.5ex minus 0.2ex} % Add some space between paragraphs
\setlength{\parskip}{0.8ex} % Add some space between paragraphs

\begin{document}

%======================================

%%%%%%%%%%%%%%%%%%%%%%%%%%%%%%%%%%%%%%%%%%%%%%%%%%%%%%%%%%%%%%%%%%%%%%%%%%%
% SPECIFIC AIMS
%%%%%%%%%%%%%%%%%%%%%%%%%%%%%%%%%%%%%%%%%%%%%%%%%%%%%%%%%%%%%%%%%%%%%%%%%%%

%======================================

\noindent \begin{center}
{\bf SPECIFIC AIMS}
\end{center}

Genetically-encoded probes have revolutionized our understanding of cellular processes. Fluorescent proteins are a powerful technology that enable tracking of protein concentration and localization over large timescales and with single-molecule resolution. Via simple genetic manipulation, virtually any protein of interest can be labeled and tracked in the cell. However, analogous techniques for study of other biomolecules have yet to be developed.

The importance of RNA in controlling a wide range of cellular functions is only recently being realized. Only (SOME FRACTION HERE) of RNA are actually involved in protein synthesis, while the rest is not well understood. This lack of understanding is due in part to the lack of tools available to image this biomolecule. Localization of RNA on a single-molecule level, and multicomponent imaging of RNA transcripts remains difficult. Existing tools utilize aptamers that are unstable in mammalian cells, or constructs that are too large for imaging small transcripts. Multicomponent RNA imaging is also difficult due to the design of current tools.

To address this need, \textit{I aim to develop a platform for RNA imaging that will enable facile tracking of multiple transcripts at high resolution}. Riboglow is an RNA imaging platform recently developed in the Palmer lab. It utilizes a fluorescence-quenched pair formed by cobalamin (vitamin $B_12$) and a pendant fluorophore. In solution, this construct shows low fluorescence. When bound to the cobalamin riboswitch aptamer domain, there is an increase in fluorescence. This tool shows promise for RNA imaging because it solves many of the problems faced by traditional RNA probes. The platform is similar to Spinach, Broccoli, and Mango in that it utilizes a small molecule-binding aptamer for localization

{\bf \underline{Aim 1.} Adapt Riboglow for superresolution imaging.}\\
blah blah.

{\bf \underline{Aim 2.} Develop mutually orthogonal Riboglow probes for multicomponent imaging.}\\
blah blah.

{\bf \underline{Aim 3.} Application??.}\\
blah blah.

Overall... wrap up here.

\bibliographystyle{nci}
%\nobibliography{sampl}
\nobibliography{sampl-r01}
\end{document}
