%!TEX TS-program = xelatex
\documentclass{F32}

\begin{document}

% FROM THE SITE:
% Are the sponsor(s’) research qualifications (including recent publications) and track record of mentoring individuals at a similar stage appropriate for the needs of the applicant?
% Is there evidence of a match between the research and clinical interests (if applicable) of the applicant and the sponsor(s)?
% If a team of sponsors is proposed, is the team structure well justified for the mentored training plan, and are the roles of the individual members appropriate and clearly defined?
% Are the research facilities, resources (e.g., equipment, laboratory space, computer time, subject populations), and training opportunities (e.g. seminars, workshops, professional development opportunities) adequate and appropriate?
% Is the institutional environment for the applicant’s scientific development of high quality?
% Is there appropriate institutional commitment to fostering the applicant’s mentored training.

\begin{center}
{\bf SELECTION OF SPONSOR AND INSTITUTION}
\end{center}

The proposed research will be carried out in the lab of Professor Amy Palmer in the Biofrontiers Institute at the University of Colorado Boulder. The Biofrontiers Institute is a world leader in both imaging and RNA research. In choosing this community to carry out my research, I am in an excellent position to be successful. RNA is a large focus of the institute. With Nobel Prize winner Tom Cech and its director, and the Rinn and Batey labs (see letters of mentoring support) in the same building, I will have easy access to the top researchers in the field of RNA. Collaboration with these labs will provide me with additional techniques and input that may not be readily available in the Palmer lab itself. The institute also provides superior research facilities and instrumentation that are maintained by full-time staff. Within the same wing as the Palmer lab is located both the cell culture facilities, and the microscopy core, and the building houses the deep sequencing and mass spectrometry cores.

I will also have a high chance of success due to my choice in mentor. Professor Palmer was educated by some of the best in the field, with her graduate degree in the lab of Edward Solomon at Stanford, and her postdoctoral work with Roger Tsien at the University of California, San Diego. She is an expert in the field of fluorescent tools development, and the study of fundamental cellular processes. When she was a postdoc in the Tsien lab, she was also awarded the Ruth L. Kirschstein National Research Service Award Postdoctoral Fellowship, thus she is aware of the level of training and mentorship that is necessary for awardees. Because my independent research goals lie in tool development at the interface of molecular organic chemistry and chemical biology, she is an excellent choice for training in this area. Additionally, professor Palmer has already mentored four group members that have gone on to acquire tenure-track professorships at research institutions.

\end{document}
