% !TEX root = ../Research_Specific_Aims_and_Bib.tex

% FROM THE NIH SITE:


\begin{center}
{\bf RESPONSE TO REVIEWERS}
\end{center}
The grant application 1F32GM131666-01 titled ``Multicomponent, superresolution imaging of RNA in mammalian cells'' was generally well received by reviewers. All rated the applicant, sponsor, and institutional environment very highly. Concerns expressed with regards to the research proposal and the training potential are addressed below, and changes were made to the corresponding documents.

\subsubsection*{Critique 1}
The first reviewer suggested several helpful changes to the proposal. They pointed out several references to aims 1 and 3 that add precedence to the proposed work. These references were added to the revision. This reviewer also asked for additional clarification regarding the signal-to-background required for the single-molecule imaging experiments proposed in Aim 2. To address this concern, details were added to this aim. Recent data acquired in the lab has shown that single molecules of beta-actin mRNA can be distinguished by linking 12 copies of Riboglow in series on a single transcript. This provides concrete evidence that in order to image RNA with a single copy of Riboglow, the system needs to be improved by at least 12 fold. Changes have been made to aim 2 to reflect these new data and clarify the improvements necessary to make Riboglow a viable single-molecule imaging probe. Finally, concern was expressed that the aptamer tag may interfere with the biosynthesis of the lncRNA to be studied. Though Riboglow has not been tested with lncRNA to date, localization of lncRNA in living cells has been carried out with the MS2 system, employing four stem loops encoded downstream of the transcript of interest \cite{YoonLincRNAp21SuppressesTarget2012}. Additionally, our target lncRNA of interest, \textit{SNHG5}, has also been tagged with the MS2 coat protein for RNA immunoprecipitation (RIP) without perturbing biosynthesis \cite{HeLncRNASNHG5regulates2017}. Thus, we expect that Riboglow, a much smaller probe than the MS2 system, should be tolerated by the lncRNA biosynthetic machinery.

The first reviewer also expressed concern with respect to the training potential of the application because the applicant is remaining in the field of imaging for their postdoctoral studies. The graduate studies of the applicant concerned imaging of whole organisms (mice) with bioluminescence, in addition to tandem modification of small heterocyclic luciferins and their matched luciferases. Though parallels are present in the Riboglow system, the study of biology on the cellular and sub-cellular level is completely new, in addition to the use of fluorescence microscopy to study these systems. The sponsor's statement has been updated to reinforce these points.

Further, the applicant's activities planned under the award have been updated to increase time spent developing independent research ideas and applying for tenure-track professorships in years two and three, as suggested by this reviewer.

\subsubsection*{Critique 2}
\comment{Some way to address NIH funding ending before award receipt?}

The second reviewer also sought additional clarification for several aspects of the proposal. The advantages of the Riboglow platform over existing technologies were unclear to them. The main advantages are two fold: Riboglow is (1) smaller and less perturbing than existing systems, and (2) more sensitive than the modern techniques. A clearer statement to this effect was added to the proposal. This reviewer also requested clarification with respect to the level of fluorescence required for single molecule imaging (similar to the first reviewer). As stated above, 12 copies of Riboglow are currently required for single molecule resolution, thus the first generation system will need to be improved by at least 12 fold to enable imaging with a single copy of the probe. This 12 fold increase will be attained through a combination of improvements to fluorescence quenching and fluorophore quantum yield (Aim 1), and optimization of fluorescence turn-on upon riboswitch binding (Aim 2).

Concern was also expressed by this reviewer regarding the professional development that the applicant will receive at CU Boulder. Additional programs and activities have been identified to augment the applicant's Goals for Fellowship Training. These include the Graduate Teacher Program, the SMART mentorship program, and additional events offered by the Office of Postdoctoral Affairs.

\subsubsection*{Critique 3}
\comment{Some way to address comment about lack of publications in six years of grad work?}

The final reviewer expressed minor concerns regarding the mentorship track record of the sponsor. Any lack in mentoring experience should be accounted for via the additional mentoring received from Professors Batey and Rinn (see support letters) who have committed to providing both research input and advise regarding professional development and career goals (R01 professorship).


%%% Local Variables: ***
%%% mode: latex ***
%%% TeX-master: "Research_and_SA.tex" ***
%%% End: ***
