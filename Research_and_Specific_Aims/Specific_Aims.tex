% !TEX root = ../Research_Specific_Aims_and_Bib.tex

% FROM THE NIH SITE:
% Is the proposed research project of high scientific quality, and is it well integrated with the proposed research training plan?
% Based on the sponsor’s description of his/her active research program, is the applicant’s proposed research project sufficiently distinct from the sponsor’s funded research for the applicant’s career stage?
% Is the research project consistent with the applicant’s stage of research development?
% Is the proposed time frame feasible to accomplish the proposed training?

\noindent \begin{center}
{\bf SPECIFIC AIMS}
\end{center}

Genetically-encoded probes have revolutionized our understanding of cellular processes. Fluorescent proteins are a powerful technology that enable tracking of protein concentration and localization over large timescales and with single-molecule resolution. Via simple genetic manipulation, virtually any protein of interest can be labeled and tracked in the cell. However, analogous techniques for study of other biomolecules through genetic encoding (e.g. RNA)  are underdeveloped. There is currently no approach suitable for tracking the diverse types of RNAs found in mammalian cells. % Amy: suggested edit: "are far less developed and there is currently no approach suitable for tracking the diverse types of RNAs found in mammalian cells."

The importance of RNA in controlling a wide range of cellular functions has only recently been realized \cite{CechNoncodingRNARevolution2014}. Signaling by messenger and noncoding RNAs in the cell is highly dependent on the location of the transcripts. It is understood that cellular machinery modulates the location of these molecules, but these processes remain difficult to elucidate \cite{Muller-McNicollHowcellsget2013a}.
%Only (SOME FRACTION HERE) of RNA are actually involved in protein synthesis, while the rest is not well understood.
This lack of understanding is due in part to the lack of tools available to image this biomolecule. Localization of RNA on a single-molecule level, and multicomponent imaging of RNA transcripts remains difficult. Existing tools utilize aptamers that are unstable in mammalian cells \cite{EtzelSyntheticRiboswitchesPlug2017}, or constructs that are too large for imaging most transcripts. Multicomponent RNA imaging is also difficult due to the design of current tools.

To address this need, \textit{I aim to develop a platform for RNA imaging that will enable facile tracking of multiple transcripts at high resolution}. Riboglow is an RNA imaging platform recently developed in the Palmer lab \cite{Braselmannmulticolorriboswitchbasedplatform2018}. It utilizes a fluorescence-quenched pair formed by cobalamin (vitamin B\textsubscript{12}) and a pendant fluorophore. In solution, this construct shows low fluorescence. When bound to the cobalamin riboswitch aptamer domain \cite{JohnsonJrB12cofactorsdirectly2012}, there is an increase in fluorescence. This tool shows promise for RNA imaging because it solves many of the problems faced by traditional RNA probes.
The platform is similar to Spinach \cite{PaigeRNAMimicsGreen2011}, Broccoli \cite{FilonovBroccoliRapidSelection2014}, and Mango \cite{AutourFluorogenicRNAMango2018,DolgosheinaRNAMangoAptamerFluorophore2014} in that it utilizes a small molecule-binding aptamer for localization, but improves on these tools because it utilizes a natural aptamer that is more stable in the cellular environment (because of its native fold).
The current gold-standard for imaging is the MS2-FP system \cite{FuscoSinglemRNAMolecules2003}, which utilizes an stem-loop-binding bacteriophage coat protein fused to a fluorescent protein. Though this technique benefits from the modularity of fluorescent proteins, it requires large constructs to concentrate the fluorescent signal (24 stem-loops are often placed in series), thus precluding its use with most RNAs.

Riboglow is already a useful tool for RNA imaging, but several drawbacks are keeping it from widespread utility. The proposed work addresses these drawbacks, and seeks to utilize improved Riboglow tools to study outstanding questions in the field of noncoding RNA. The \underline{\textit{objective}} of the work is to image noncoding RNAs as they function within living cells. My \underline{\textit{central hypothesis}} is that the Riboglow platform can be improved through chemical modification and RNA selection to provide brighter, multicomponent probes.

{\bf \underline{Aim 1.} Synthesize improved Riboglow probes.}\\
Previously developed Riboglow constructs suffered from poor signal induction and low brightness. I aim to derivatize the native cobalamin structure to produce new probe scaffolds. The linker and fluorophore will also be varied with the goal to maximize fluorescence turn-on. These new molecules will be evaluated for quenching efficiency and will be tested with native cobalamin aptamers.

{\bf \underline{Aim 2.} Adapt Riboglow for single-molecule imaging.}\\
The molecules developed in \textit{Aim 1} will be screened against libraries of aptamers to further improve probe properties. Screening will be carried out in mammalian cells via flow cytometry. Candidate probes will be verified through imaging of mRNA transcripts in living cells.

{\bf \underline{Aim 3.} Develop mutually orthogonal Riboglow probes for multicomponent imaging.}\\
In tandem with brightness optimization, I will develop mutually orthogonal probes to enable labeling of different RNA transcripts in the same cell. The power of SELEX to find selective and tight binders will be used to screen for mutually exclusive aptamer-cobalamin pairs. These pairs will be conjugated to spectrally-resolved fluorophores to enable tracking of multiple RNA simultaneously. These orthogonal probes will be used to image long noncoding RNA and mRNA as they interact in the cell.

%%% Local Variables: ***
%%% mode: latex ***
%%% TeX-master: "Research_and_SA.tex" ***
%%% End: ***
