%!TEX TS-program = xelatex
\documentclass{F32}

\begin{document}


\begin{center}
{\bf INSTITUTIONAL ENVIRONMENT AND COMMITMENT TO TRAINING}
\end{center}

\begin{enumerate}
  \item \textbf{The University of Colorado Boulder:} University of Colorado Boulder is home to the Postdoctoral Association of Colorado. This group provides a regular mailing list of announcements for a wide variety of professional development and funding opportunities. They also organize regular events with a focus on career development, networking, social activities and advocacy. The Office of Postdoctoral Affairs is another avenue of support on campus for postdoctoral training and career development. They offer a number of services to CU postdocs such as networking events, seminars, and tutorials. Postdoctoral researchers are also welcome to attend events hosted by the graduate school at CU Boulder, including workshops, lectures, and networking events. Postdoctoral researchers are also invited to participate in an array of scientific communication forums, such as serving as speakers for Supergroup meetings and at the Biochemistry annual retreat.

  \item \textbf{The BioFrontiers Institute:} The scientific environment within CU Boulder and the BioFrontiers Institute is excellent. Professor Palmer is a part of both the Department of Chemistry and Biochemistry, and the BioFrontiers Institute, directed by Nobel Laureate Tom Cech. The institute has become a leading organization for RNA research, with both highly established faculty such as Robert Batey (see support letter), Roy Parker (HHMI Investigator), and Tom Cech, and newcomers like John Rinn who has recently moved from Harvard (see support letter). Not only are all these labs within the same building (JSCBB), they also meet regularly as supergroups to talk about current research, and to find opportunities to collaborate. The Palmer lab participates in the Biophysics, Signaling and Cellular Regulation, Chemical Biology, RNA, and Bioinformatics supergroups which meet bi-weekly.

  \item \textbf{The Palmer Lab:} Professor Palmer implements a lab structure that fosters an environment of collaboration and mentorship. She meets one-on-one with each student and postdoc every two weeks to discuss data, their research project, and career goals. Professor Palmer and I will also use these meeting times to \textit{(a)} discuss my future research program, \textit{(b)} grant writing and proposal construction, and \textit{(c)} mentoring opportunities and strategies.
  The Palmer lab also holds weekly group meetings in which members of the lab present their recent findings. Once a month the lab holds a special topics meeting to discuss important practical matters surrounding research. Topics include data storage and management, conduct of research, responsible image analysis, and manuscript writing.
  Within the lab itself, benches and desks are arranged together, with an abundance of shared workspace that promotes conversation and collaboration amongst labmembers.

  \item \textbf{Additional Individual Mentorship:} In addition to mentorship under Professor Palmer, I will meet regularly with professors Rinn and Batey (see letters of support) and attend both Rinn lab and Batey lab group meetings. Full Professor Batey has been a professor at CU Boulder for 15 years, and is an expert in riboswitch biology and engineering. His advise and mentorship will be instrumental for Aims 2 and 3 of my proposal. Full Professor Rinn only recently moved to CU Boulder from Harvard. He is an authority on long noncoding RNA, and will be instrumental in the experimental design regarding the use of our new tools for imaging these understudied RNAs. Taken together, professors Palmer, Batey, and Rinn form the ideal mentoring team for a postdoc that is seeking to advance to a tenure-track research professorship. Each will bring unique viewpoints and resources that will ensure my success if funded.

\end{enumerate}


\end{document}
