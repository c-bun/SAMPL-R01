%!TEX TS-program = xelatex
\documentclass{F32}

\begin{document}
\begin{center}
{\bf PROJECT SUMMARY}
\end{center}

RNA lies at the center of cellular function. Its most appreciated role is to carry protein blueprints to the ribosome for manufacture.
Only recently have researchers begun to appreciate its myriad of other functions, many of which are implicated in a variety of disease states.
Long noncoding RNA (lncRNA) comprise one such important class of RNA that does not participate in the central dogma. Alarmingly, the human genome encodes for as many lncRNA as proteins.
These transcripts are typically greater than 200 bases, and are known to participate in binding both proteins and nucleic acids, often both at the same time.
However, little else is understood regarding their function.
Where and when do they interact with their targets?
How long do these interactions occur, and what other cellular machinery is present?
This lack of understanding is due in part to the lack of tools available to image this biomolecule. Localization of RNA on a single-molecule level, and multicomponent imaging of RNA transcripts remains difficult. Existing tools utilize aptamers that are unstable in mammalian cells, or constructs that are too large for imaging small transcripts. Multicomponent RNA imaging is also difficult due to the design of current tools.

To address this need, I aim to develop a platform for RNA imaging that will enable facile tracking of multiple transcripts at single cell resolution. Riboglow is an RNA imaging platform recently developed in the Palmer lab. It utilizes a fluorescence-quenched pair formed by cobalamin (vitamin B\textsubscript{12}) and a pendant fluorophore. In solution, this construct shows low fluorescence. When bound to the cobalamin riboswitch aptamer domain, there is an increase in fluorescence. This tool shows promise for RNA imaging because it solves many of the problems faced by traditional RNA probes, however several drawbacks are keeping it from widespread utility. The proposed work addresses these drawbacks, and seeks to utilize improved Riboglow tools to study outstanding questions in the field of noncoding RNA.

Previously developed Riboglow constructs suffered from poor signal induction and low brightness. First, I aim to derivatize the native cobalamin structure, linker and fluorophore with the goal of maximizing fluorescence turn-on. These new molecules will be evaluated for quenching efficiency and signal induction. Next, the molecules I develop will be screened against libraries of riboswitch aptamers to further improve probe properties. Screening will be carried out in mammalian cells via flow cytometry, a specialty of the Palmer lab. Candidate probes will be verified through single-molecule imaging of mRNA in living cells. In tandem with brightness optimization, I will develop mutually orthogonal probes to enable labeling of different RNA transcripts in the same cell. The power of SELEX to find selective and tight binders will be used to screen for mutually exclusive aptamer-cobalamin pairs. These pairs will be conjugated to spectrally-resolved fluorophores to enable tracking of multiple RNA simultaneously. These orthogonal probes will be used to image lncRNA and mRNA as they interact in the cytosol.

\end{document}
