\documentclass[11pt]{article}
\usepackage[top=0.5in, bottom=0.5in, left=0.5in, right=0.5in]{geometry}
\usepackage{helvet}
\usepackage{url} % hypderref?
\usepackage{graphicx}
\graphicspath{{figures/}} % The figures are in a figures/ subdirectory.
\renewcommand{\familydefault}{\sfdefault}
\pagestyle{empty}
%\pagestyle{plain}

\usepackage{setspace}

\usepackage{amsfonts}
\usepackage{amsmath}

\usepackage{sidecap}
\usepackage[abs]{overpic}
\usepackage{wrapfig}

%\usepackage[round,authoryear]{natbib}
\usepackage{cite}
%\setlength{\bibsep}{0.00in}

\usepackage{hyperref}
\hypersetup{colorlinks=true, urlcolor=black, citecolor=black, linkcolor=black}

\newcommand{\doi}[1]{\href{http://dx.doi.org/#1}{doi:#1}}

\newcommand{\ac}[1]{{\sc \lowercase{#1}}}

\renewcommand{\baselinestretch}{.93}
%\renewcommand{\baselinestretch}{.90}
\usepackage{wrapfig} 

\usepackage{bibspacing}
\setlength{\bibspacing}{\baselineskip}


\graphicspath{{figs/}}

\makeatletter

\newcommand{\captionfonts}{\footnotesize}

\makeatletter  % Allow the use of @ in command names
\long\def\@makecaption#1#2{%
  \vskip\abovecaptionskip
  \sbox\@tempboxa{{\captionfonts #1: #2}}%
  \ifdim \wd\@tempboxa >\hsize
    {\captionfonts #1. #2\par}
  \else
    \hbox to\hsize{\hfil\box\@tempboxa\hfil}%
  \fi
  \vskip\belowcaptionskip}      
\makeatother

\renewcommand{\figurename}{Fig.}

% Page numbering.
%\pagestyle{plain}
%\pagenumbering{arabic}

\setlength{\abovecaptionskip}{-5pt}

\makeatother

\renewcommand{\refname}{Bibliography and References Cited}

\setlength{\parindent}{0pt} % Don't indent first line
\setlength{\parskip}{1ex plus 0.5ex minus 0.2ex} % Add some space between paragraphs

\begin{document}

%======================================

%%%%%%%%%%%%%%%%%%%%%%%%%%%%%%%%%%%%%%%%%%%%%%%%%%%%%%%%%%%%%%%%%%%%%%%%%%%
% HEADING
%%%%%%%%%%%%%%%%%%%%%%%%%%%%%%%%%%%%%%%%%%%%%%%%%%%%%%%%%%%%%%%%%%%%%%%%%%%

\begin{centering}
{\bf PROJECT NARRATIVE}

\end{centering}

% Using no more than two or three sentences, describe the relevance of this research training program to public health. In this section, use plain language that can be understood by a general, lay audience.
Physical methods for designing small molecule therapeutics are poised for a breakthrough, allowing molecular design for targeted treatment of diseases, personalized medicine, and rapid drug development. 
However, careful stress testing and improvement of these methods is necessary to make them sufficiently reliable and robust for the enormous range of problems they can potentially solve. 
Here, we will generate new experimental data on carefully selected and tailored systems, using it to drive a series of blind community challenges which follow a crowdsourcing model for testing and improving these methods, inducing new innovations and preparing the methods to have the dramatic impacts on human health that they promise. 

%======================================

%\setlength{\bibsep}{0.000in}

%\bibliographystyle{gec_nih}
%\bibliographystyle{ieeetr}
%\bibliographystyle{myrefstyle}
%\bibliography{chodera-research}


\end{document}






