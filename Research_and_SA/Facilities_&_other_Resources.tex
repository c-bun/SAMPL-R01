% !TEX root = ./Research_and_SA.tex

% FROM THE SITE:


\begin{center}
\bf FACILITIES AND OTHER RESOURCES
\end{center}
More than adequate facilities for conducting the proposed research are available in the sponsor's laboratory, and in other core facilities across the Colorado University Boulder campus. These facilities, in addition to the scientific environment at CU Boulder, are more than sufficient to ensure the success of the proposed research. \textit{Major equipment in the sponsor's laboratory is described in the Equipment form.}

  {\bf The Jennie Smoly Caruthers Biotechnology Building (JSCBB):} Dr. Palmer’s laboratory is located in the new state-of-the-art Jennie Smoly Caruthers Biotechnology Building (JSCBB). JSCBB is a four-story, 330,000 square foot research building that houses more than 60 faculty members from the BioFrontiers Institute, the Department of Chemical and Biological Engineering, and the Division of Biochemistry. The building contains most of the core facilities that are crucial to the proposed research, in addition to the labs of professors Rinn and Batey, who will be offering mentoring support (see support letters).

  {\bf The Palmer Laboratory:} The Palmer lab is located on the third floor of JSCBB and covers approximately 3,000 feet of lab space shared with the Liu Lab. The lab occupies 5 research bays, each with 3--4 benches and 3 desk spaces. Located directly across the hall is \textit{(i)} the cell culture core facility, with all necessary equipment for mammalian cell culture, and \textit{(ii)} the imaging core facility that houses a variety of state-of-the-art microscopes. A cold room, autoclave and dishwashing room, dark rooms, and shared instrument rooms are located near the lab in JSCBB as well.

  {\bf BSL-2 Certified Cell Culture Core Facility:} Located directly across the hall from the Palmer lab is a fully-equipped mammalian cell culture facility with 12 biosafety cabinets, 24 double door incubators, automated liquid nitrogen storage units, and appropriate equipment (centrifuges, light microscopes, etc). The facility is BL-2 plus certified, enabling work with viruses and pathogens. The facility is staffed by Theresa Nahreini, who provides training and support.

  {\bf BioFrontiers Advanced Light Microscopy Core:} Also located across the hall from the Palmer lab, the Advanced Light Microscopy Core houses a number of advanced microscopes: a Nikon A1R resonant laser scanning confocal microscope with perfect focus, full environment chamber for long term imaging, and equipped with TIRF; a Nikon spinning disc confocal microscope with seven laser lines, equipped with a fully enclosed environmental chamber; an Image Express high throughput high content microscope; and a Nikon N-STORM system for super-resolution (more details on specific microscope capabilities in Equipment statement). An image analysis station is available with software including MATLAB, Imaris, Nikon Elements, ImageJ/Fiji, ICY, Cell Profiler. The facility is staffed by core director Dr. Joe Dragavon, who offers  training and assisted imaging sessions. Professor Palmer is the faculty advisor.

  {\bf BioFrontiers Next-Generation Genomics Facility:} Located in JSCBB, this facility houses an Illumina HiSeq 2000 sequencer, two MiSeq sequencers and associated Illumina iCompute infrastructure for analysis of sequencing data. Staff members aid in library construction and assessment of sequencing quality.

  {\bf Biomolecular Mass Spectrometry and Proteomics:} Also located in JSCBB, this facility specializes in both macromolecular and small-molecule mass spectrometry. The facility houses a LTQ-Orbitrap, and Orbitrap Velos mass spectrometer with CID/HCD/ETD fragmentation capabilities, interfaced with a Waters NanoAcquity 2D-UPLC system. Other instruments include a LTQ-Orbitrap mass spectrometer with CID fragmentation capabilities (used for other proteomics), interfaced with a Waters NanoAcquity 2D-UPLC system, a Waters Synapt G2 QqTOF mass spectrometer with Waters Acquity UPLC system (mainly used for small molecule and hydrogen-deuterium exchange studies), a PerSeptive Voyager DE-STR MALDI-TOF mass spectrometer, and a AB 4000 QTrap Linear Ion Trap mass spectrometer with Eksigent 2D-LC nanoflow HPLC. For small molecule analysis, the facility offers accurate mass determinations and complex mixture analysis.

  {\bf NMR Spectroscopy Facility:} In JSCBB, a Bruker Avance-III NMR spectrometer operating at 400MHz for proton NMR is available for walk-up use. It is capable of one and two-dimensional NMR experiments on most NMR-active nucleii, and is equipped with a 60-slot autosampler. This instrument is complemented by a larger facility nearby on campus containing a similar 300MHz instrument, a 400MHz Varian INOVA 400, and a 500 MHz Varian INOVA 500.

  {\bf Cell Sorting Facility:} Located directly across the hall from the Palmer Lab, the cell sorting facility contains a FACSAria III (BD Biosciences), and a MoFlo cell sorter from Cytomation. The facility, and training for its use is available to all labs in the building.

  {\bf Computers:} Several computers and data storage servers are available in the Palmer lab. The Biochemistry division of the Chemistry department also offers DEC and SUN workstations, and Silicon Graphics terminals for molecular modeling.

  {\bf The JSCBB Shared Instruments Pool:} JSCBB also houses a shared instruments pool that provides access to a variety of analytical instruments necessary for biochemical studies. The facility contains EPR, CD, and fluorescence spectrometers, an isothermal titration calorimeter, and other instruments.

  {\bf Institutional Postdoctoral Support and Resources:} \textit{See document ``Institutional Environment and Commitment to Training'' document.}

%%% Local Variables: ***
%%% mode: latex ***
%%% TeX-master: "Research_and_SA.tex" ***
%%% End: ***
