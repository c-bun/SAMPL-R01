% !TEX root = ./Research_and_SA.tex

\textbf{SIGNIFICANCE}
RNA lies at the center of cellular function. Its most appreciated function is to carry protein blueprints from the genetic code of the nucleus to be manufactured into the protein machines of the cell at the ribosome. Only recently have researchers begun to appreciate its myriad of other functions, many of which also lie at the center of important cellular processes. \comment{X, Y, and Z (maybe the proposed test systems??)} have been recently found to have crucial RNA components. This importance for cellular function necessitates a toolset of probes to study these RNAs as they traverse the cell. Indeed, the location of these RNA transcripts has been implicated to be instrumental in their function.\cite{Muller-McNicollHowcellsget2013a}

Proteins benefit from a large imaging toolkit that has been developed over the last \comment{XX} years. Genetically encodable fluorescent proteins are now ubiquitous for the study of the localization of any translated target in the cell.

%%%%%%%%%%%%%%%%%%%%%%%%%%%%%%%%%%%%%%%%%%%%%%%%%%%%%%%%%%%%%%%%%%%%%%%%%%%%%%%%
%Riboglow
\begin{wrapfigure}[18]{r}{10cm}
%\vspace{-0.2in}
\begin{centering}
\includegraphics[width=\textwidth]{figures/fig1.pdf}

\end{centering}
\footnotesize
\caption{\label{figure:riboglow}
A) Cobalamin acts as a quenching and localization moiety to guide a fluorescent probe to an RNA transcript of interest. When unbound, fluorescence is quenched. In the presence of RNA tagged with the cobalamin aptamer, fluorescence is restored.
}
\end{wrapfigure}
%%%%%%%%%%%%%%%%%%%%%%%%%%%%%%%%%%%%%%%%%%%%%%%%%%%%%%%%%%%%%%%%%%%%%%%%%%%%%%%%

\textbf{BACKGROUND}
Lorem ipsum dolor sit amet, consectetur adipisicing elit, sed do eiusmod tempor incididunt ut labore et dolore magna aliqua. Ut enim ad minim veniam, quis nostrud exercitation ullamco laboris nisi ut aliquip ex ea commodo consequat. Duis aute irure dolor in reprehenderit in voluptate velit esse cillum dolore eu fugiat nulla pariatur. Excepteur sint occaecat cupidatat non proident, sunt in culpa qui officia deserunt mollit anim id est laborum.

Lorem ipsum dolor sit amet, consectetur adipisicing elit, sed do eiusmod tempor incididunt ut labore et dolore magna aliqua. Ut enim ad minim veniam, quis nostrud exercitation ullamco laboris nisi ut aliquip ex ea commodo consequat. Duis aute irure dolor in reprehenderit in voluptate velit esse cillum dolore eu fugiat nulla pariatur. Excepteur sint occaecat cupidatat non proident, sunt in culpa qui officia deserunt mollit anim id est laborum.

\textbf{APPROACH \underline{Aim 1.} Synthesize improved Riboglow probes.}\\
The main drawback of Riboglow is the poor turn-on that is observed upon probe binding. In this aim, I intend to leverage my background in synthetic chemistry to produce a panel of diverse probe structures that improve fluorescence quenching (and thus signal induction). In previous studies in collaboration with Professor Dorota Gryko (see Gryko letter of support) a small number of linkers and fluorophores were evaluated for quenching and fluorescence turn-on. Linker length and fluorophore wavelength were varied to gauge the quenching ability of cobalamin. Remarkably some degree of quenching occurred in all of the constructs synthesized, regardless of the spectral overlap of the fluorophore and the cobalamin. The largest amount of quenching was observed in the probes with the shortest linkers, and the largest spectral overlap. \textit{To optimize probe function, I will vary linker composition,  cobalamin metal environment, and pendant fluorophore.}

Ideally, in the unbound state, the cobalamin and the fluorophore would be closely associated to maximize FRET and contact quenching.\cite{LeeDesignSynthesisCharacterization2009} In the bound state, the molecules would reside at their maximal distance to promote fluorescence. To strike this balance, I propose the use of a synthetic beta turn as the linker between cobalamin and the fluorophore. Such a linker would hold the molecules close in solution, but would be linearized upon binding to the aptamer. A number of such beta turns have been developed. These motifs are as small as twelve amino acids and many are stable to denaturation up to 85 C.\cite{KierProbingLowerSize2008} In the unbound state, such a linker would hold the quencher and fluorophore in close proximity (due to the short distance between the N and C termini of the peptide). When the cobalamin is bound by an aptamer, steric occlusion would force the beta turn to unfold to place the fluorophore-quencher pair at a larger distance. The amino acids of the peptide linker will be varied to adjust the stability of the fold.
% TODO make this less specific?

Fluorescence turn-on will also be modulated through changes to the cobalamin metal center. Changes to the electronic environment of the metal center will shift the absorption spectrum, enabling greater spectral overlap with the fluorophore to be quenched.\comment{[Cite]} As the \textit{de novo} synthesis of a molecule such as vitamin B12 would be a massive undertaking,\comment{[Cite]} I will target modifications that can be made through derivatization of the native structure. Without modification of the native ligand, variation of the axial position of cobalamin, and of the metal center itself should be straightforward. There is a large body of work that targets such modifications for the synthesis of so called ``antivitamins".\cite{KrautlerBernhardAntivitaminsB12Structure2015,ChrominskiReductionfreesynthesisstable2013} Through this undertaking, I will be in contact with Professor Dorota Gryko (see Gryko letter of support) one of the experts in this field.
% TODO will changing the axial ligand actually change the absorption??
% TODO will changing the metal change the absorption?
% TODO make a point here about how unnatural ligand architectures will help reduce off-target fluorescence.

The final variable in the molecular structure of the Riboglow probe is the fluorophore itself. Previous work showed that probes were quenched by the cobalamin center to varying degrees. Quenching correlated somewhat with the spectral overlap of the fluorophore and cobalamin absorption.
% TODO is this 100% true? compare Cy5 excition/emission with ATTO probes.
% TODO make a point here about how redder is better

The new Riboglow probe constructs that I synthesize will be evaluated for brightness in the presence and absence of the cobalamin aptamer.

\textbf{\underline{Aim 2.} Adapt Riboglow for superresolution imaging.}\\
Visualization of the lifecycle of single RNA transcripts as they move throughout the cell remains a holy grail of RNA imaging.\comment{[Cite]} Such a goal should be possible via superresolution imaging and a probe with adequate photostability. Such requirements should be achievable with the Riboglow platform. With a variety of new probe constructs in hand, changes will be made to the cobalamin aptamer to increase fluorescence turn-on. The Palmer lab is a leader in technologies for tool development in mammalian cells.\comment{[Cite]} This expertise will be leveraged for screening libraries of cobalamin aptamers in mammalian cells. Libraries of transcripts will be transduced into mammalian cells, the probe of interest will be administered, and cells will be sorted via flow cytometry. 
% QUESTION relative to an FP expression control? or some other RNA-based technique? Broccoli?
In this way, libraries of up to one million members will be screened. Bright variants will be collected, cultured, and resubjected to sorting until the library converges. Sequences will be evaluated through deep sequencing.

%%%%%%%%%%%%%%%%%%%%%%%%%%%%%%%%%%%%%%%%%%%%%%%%%%%%%%%%%%%%%%%%%%%%%%%%%%%%%%%%
%Multicomponent
\begin{wrapfigure}[25]{l}{10cm}
%\vspace{-0.2in}
\begin{centering}
\includegraphics[width=\textwidth]{figures/fig3.pdf}

\end{centering}
\footnotesize
\caption{\label{figure:multicomponent}
A) Mutually orthogonal cobalamin analogs will enable multicomponent RNA imaging. Signal turn-on will only be observed in the presence of the matched pair. B)\comment{Propose some structures.} C) SELEX will be used to screen for aptamers that bind each cobalamin in a mutually-exclusive manner.
}
\end{wrapfigure}
%%%%%%%%%%%%%%%%%%%%%%%%%%%%%%%%%%%%%%%%%%%%%%%%%%%%%%%%%%%%%%%%%%%%%%%%%%%%%%%%

\textbf{\underline{Aim 3.} Develop mutually orthogonal Riboglow probes for multicomponent imaging.}\\
Lorem ipsum dolor sit amet, consectetur adipisicing elit, sed do eiusmod tempor incididunt ut labore et dolore magna aliqua. Ut enim ad minim veniam, quis nostrud exercitation ullamco laboris nisi ut aliquip ex ea commodo consequat. Duis aute irure dolor in reprehenderit in voluptate velit esse cillum dolore eu fugiat nulla pariatur. Excepteur sint occaecat cupidatat non proident, sunt in culpa qui officia deserunt mollit anim id est laborum.

%%% Local Variables: ***
%%% mode: latex ***
%%% TeX-master: "Research_and_SA.tex" ***
%%% End: ***
