% !TEX root = ./Research_and_SA.tex

% FROM THE SITE:
% the reviewers will evaluate the adequacy of the proposed RCR training in relation to the following five required components:
% 1) Format - the required format of instruction, i.e., face-to-face lectures, coursework, and/or real-time discussion groups (a plan with only on-line instruction is not acceptable);
% 2) Subject Matter - the breadth of subject matter, e.g., conflict of interest, authorship, data management, human subjects and animal use, laboratory safety, research misconduct, research ethics;
% 3) Faculty Participation - the role of the sponsor(s) and other faculty involvement in the fellow’s instruction;
% 4) Duration of Instruction - the number of contact hours of instruction (at least eight contact hours are required); and
% 5) Frequency of Instruction – instruction must occur during each career stage and at least once every four years.

\begin{center}
\bf RESPONSIBLE CONDUCT OF RESEARCH
\end{center}

My graduate studies at UC Irvine included significant early training in the conduct of research that has laid a foundation for the rest of my career. During my first year of study at UCI, I took a ``Conduct of Research'' class from Professor Dave Van Vranken. The class met three times a week for a quarter (30 hours), and discussed a range of issues relevant to research ethics at an academic institution. Additionally, on a quarterly basis throughout my graduate career, I met with my research advisor, Professor Jenn Prescher, to discuss my conduct in the lab, the clarity of my lab notebook, mentoring, and authorship. I remain committed to the highest standard of research conduct, and look forward to receiving additional training in this area as a postdoctoral researcher. During my time at the University of Colorado, I will be taking ``Scientific Ethics and the Responsible Conduct of Research'' from Professor Mary Allen. I have outlined the instruction I will be receiving in this class below.

\begin{enumerate}
  \item{\bf Format}

  On a weekly basis the class will meet to hear a lecture from a member of the CU Boulder research faculty on a relevant topic on conduct of research. Immediately following each lecture, the class will break into small groups for discussion. Between classes, there will be assigned readings with accompanying questions to facilitate class discussion. Over the course of the semester, two case study essays will be written, reporting on real issues that researchers faced regarding one of the topics in the course.
  \item{\bf Subject Matter}

  The class reading will be from ``Responsible Conduct of Research'' by Adil Shamoo and David Resnik. (3rd ed. 2015, Oxford University Press)\\

  \textit{Topics will include:}
  \begin{enumerate}
  \item{Why does RCR matter?}
  \item{The Scientist in Society}
  \item{Mentor/Trainee Issues}
  \item{Data Acquisition, Management, and Reproducibility}
  \item{Lab Safety}
  \item{Authorship, Publication, and Peer Review}
  \item{Conflict of Interest}
  \item{Research Misconduct}
  \item{Protection of Human Subjects}
  \item{Collaborative Research}
  \item{Intellectual Property}
  \end{enumerate}
  \item{\bf Faculty Participation}

  In addition to interactions with faculty speakers during class, I will also meet with Professor Palmer on a biweekly basis to discuss responsible conduct of research during my first semester in the lab (Fall 2018). These meetings will cover data acquisition and storage, mentoring of graduate and undergraduate students, authorship, and collaboration in the Palmer lab. The lab also conducts monthly meetings on special research topics including data storage and management, conduct of research, responsible image analysis, and manuscript writing.
  \item{\bf Duration of Instruction}

  The class will meet on a weekly basis during the spring semester of 2019 for 15 weeks. Each class is two hours, giving a total of 30 contact hours.
  \item{\bf Frequency of Instruction}

  I will begin my postdoctoral studies in the fall of 2018, and enroll in the Responsible Conduct of Research class to be offered in the spring of 2019. This will fulfill the requirement for training during my postdoctoral career.
\end{enumerate}

%%% Local Variables: ***
%%% mode: latex ***
%%% TeX-master: "Research_and_SA.tex" ***
%%% End: ***
